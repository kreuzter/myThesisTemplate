\chapter*{Seznam symbolů použitých v~textu}
\label{chap:seznam}
\begin{longtable}[H]{p{1.5cm}|p{2.25cm}|p{4.25cm}x{.5cm}|p{5cm}}
%\begin{longtable}[H]{l|l|lr|l}

 \textbf{značka}& \textbf{jednotka}   &\textbf{definiční vztah}&& \textbf{název} \tn   \hline \hline

 A~             & [m$^{2}$]           &&& plocha  \tn  
 A$_{TS}$       & [m$^{2}$]           &&& teplosměnná plocha  \tn  
 c              & [J (kg K)$^{-1}$]   &&& měrná tepelná kapacita \tn 
 C$_f$          & [-]                 &&& Fanningův součinitel třecích ztrát \tn   
 D$_T$          & [m$^{2}$ s$^{-1}$]  &&& difuzní koeficient \tn 
 D              & [m]                 &&& charakteristický rozměr  \tn 
 d              & [m]                 &&& průměr střednice šroubovice  \tn
 De             & [-] & $\mathrm{De} = \mathrm{Re} \sqrt{\frac{D_h}{d}}$ &\Debor & Deanovo číslo  \tn 
 D$_h$          & [m] & $D_h = \frac{4A}{o}$ &\Debor& hydraulický průměr  \tn
 f              & [-]                 &&& Darcyho součinitel třecích ztrát \tn 
 g              & [ms$^{-2}$]         &&& gravitační zrychlení \tn 
 Gr             & [-] &$\mathrm {Gr} =\ddfrac {g\beta (T_{w}-T_{\infty })D^{3}}{\nu ^{2}}$ &\Debor\footnote{Význam znaku \Debor \ je vysvětlen na str. \pageref{lines:Debor}}& Grashofovo číslo\tn 
 h              & [Wm$^{-2}$K$^{-1}$] &&& součinitel přestupu tepla \tn  
 I              & [A]                 &&& elektrický proud \tn {}
 l              & [m]                 &&& délka \tn  
 Nu             & [-] &$\mathrm{Nu}=\ddfrac {h \cdot D}{\lambda }$ &\Debor& Nusseltovo číslo\tn
 o              & [m]                 &&& (smočený) obvod \tn       
 Pe             & [-] &$\mathrm{Pe}=\ddfrac {D u~}{\alpha}$ && Pécletovo číslo\tn 
 Pr             & [-] &$\mathrm{Pr}=\ddfrac {\nu }{\alpha}=\ddfrac {c_p \mu }{\lambda}$ &\Debor& Prandtlovo číslo\tn 
 Q              & [J]                 &&& teplo \tn 
 R              & [$\Omega$]          &&& odpor \tn  
 Ra             & [-] &$\mathrm{Ra}=\ddfrac {\rho \beta (T_{w}-T_{\infty }) D^{3}g}{\mu  \alpha }$ &\Debor& Rayleigho číslo\tn 
 Re             & [-] &$\mathrm{Re}=\ddfrac{u \cdot D}{\nu }$ &\Debor& Reynoldsovo číslo \tn 
 S$_T$          & [K s$^{-1}$]        &&& zdroj v~rovnici vedení tepla \tn  
 T              & [K]                 &&& termodynamická teplota  \tn 
 T$_w$          & [K]                 &&& teplota na stěně  \tn
 T$_\infty$     & [K]                 &&& teplota v~dostatečné vzdálenosti od zkoumaného tělesa  \tn
 $\Delta$ T$_{\ln}$  & [K] & $\Delta T_{\ln}={\frac {\Delta T_{A}-\Delta T_{B}}{\ln \left({\frac {\Delta T_{A}}{\Delta T_{B}}}\right)}}$ && střední logaritmický spád \tn 
 t              & [s]                 &&& čas \tn 
 u~             & [ms$^{-1}$]         &&& rychlost \tn 
 V              & [m$^{3}$]           &&& objem  \tn  \hline
 
 $\alpha$       & [m$^{2}$s$^{-1}$]    &&& tepelná difuzivita \tn   
 $\alpha$       & [K$^{-1}$]           &&& součinitel délkové roz\-taž\-nos\-ti  \tn    
 $\beta$        & [K$^{-1}$]           &&& součinitel objemové roz\-taž\-nos\-ti \tn 
 $\Delta$ p     & [Pa]                 &&& tlakové ztráty \tn 
 $\epsilon$     & [-]                  &&& zářivost \tn     
 $\epsilon$     & [m]                  &&& drsnost povrchu \tn
 $\lambda$      & [[Wm$^{-}$K$^{-1}$]] &&& tepelná vodivost \tn 
 $\mu$          & [N s~m$^{-2}$]       &&& dynamická viskozita \tn      
 $\nu$          & [m$^{2}$ s$^{-1}$]   &$\nu ={\ddfrac {\mu }{\rho }}$& \Debor& kinematická viskozita \tn      
 $\rho$         & [kg m$^{-3}$]        &&& hustota \tn   
 $\rho_E$       & [$\Omega$ m]         &&& resistivita \tn   
 $\tau_w$       & [Pa] &$\tau_w = \mu \left. {\ddfrac {\partial u}{\partial y}} \right|_{y=0}$ && smykové napětí u~stěny \tn \hline
\end{longtable}

\subsubsection*{Seznam indexů a~zkratek}
\begin{longtable}[H]{r|l}

                  & \textbf{význam}   \\  \hline
    AE            & analyticko-empirický \\
    krit          & kritický          \\
    EM            & elektromotor      \\
    ref           & referenční        \\
    SEM           & synchronní EM     \\  
\end{longtable}