\newgeometry{left=0.1cm,bottom=0.1cm, right=0.1cm, top=0.1cm}
\thispagestyle{empty} \addtocounter{page}{-1}
.\newpage
\thispagestyle{empty} \addtocounter{page}{-1}
\begin{figure}[H]
      \centering
      \includegraphics[width=1\textwidth]{Images/zav_prace.pdf}
\end{figure}
\newpage
.
\restoregeometry

% přesně podle formuláře "Zadání bak./dipl. práce" VYPLŇTE:
\newcommand{\cvut}{České vysoké učení technické v~Praze}
\newcommand{\fs}{Fakulta strojní}
\newcommand{\um}{Ústav mechaniky tekutin a~termodynamiky}
\newcommand{\program}{Aplikované vědy ve strojním inženýrství} % změňte, pokud máte jiný stud. program
\newcommand{\obor}{Aplikovaná mechanika} % změňte, pokud máte jiný obor

\newcommand{\druh}{Diplomová práce} % nebo "Diplomová práce"
\newcommand{\nazevcz}{český název práce (přesně podle zadání!)}    % český název práce (přesně podle zadání!)
\newcommand{\nazeven}{anglický název práce (přesně podle zadání!)}          % anglický název práce (přesně podle zadání!)
\newcommand{\autor}{Autor}   % vyplňte své jméno a~příjmení (s akademickým titulem, máte-li jej)
\newcommand{\vedouci}{vedoucí} % vyplňte jméno a~příjmení vedoucího práce, včetně titulů, např.: Doc. Ing. Ivo Malý, Ph.D.
\newcommand{\pracovisteVed}{\um, \\ \fs, \cvut} % ZMĚŇTE, pokud vedoucí Vaší práce není z~KSI
\newcommand{\konzultant}{--} % POKUD MÁTE určeného konzultanta, NAPIŠTE jeho jméno a~příjmení
\newcommand{\pracovisteKonz}{--} % POKUD MÁTE konzultanta, NAPIŠTE jeho pracoviště

% podle skutečnosti VYPLŇTE:
\newcommand{\rok}{rok}  % rok odevzdání práce (jen rok odevzdání, nikoli celý akademický rok!)
\newcommand{\kde}{Praze} % studenti z~Děčína ZMĚNÍ na: "Děčíně" (doplní se k~"prohlášení")

\newcommand{\klicova}{klíčová, slova, se oddělují, čárkou}   % zde NAPIŠTE česky max. 5 klíčových slov
\newcommand{\keyword}{key, words}       % zde NAPIŠTE anglicky max. 5 klíčových slov (přeložte z~češtiny)
\newcommand{\abstrCZ}{abstrakt abstrakt
}    % zde NAPIŠTE abstrakt v~češtině (cca 7 vět, min. 80 slov)
\newcommand{\abstrEN}{abstrakt abstrakt} % zde NAPIŠTE abstrakt v~angličtině

% příprava:    (na následujících 8 řádků NESAHEJTE!)
\newbox\odstavecbox
\newlength\vyskaodstavce
\newcommand\odstavec[2]{%
    \setbox\odstavecbox=\hbox{%
         \parbox[t]{#1}{#2\vrule width 0pt depth 4pt}}%
    \global\vyskaodstavce=\dp\odstavecbox
    \box\odstavecbox}
\newcommand{\delka}{105mm} % šířka textů ve 2. sloupci tabulky

% použití přípravy:    % dovnitř "tabular" vůbec NESAHEJTE!
\begin{tabular}{ll}
  {\em Název práce:} & ~ \\
  \multicolumn{2}{l}{\odstavec{\textwidth}{\bf \nazevcz}} \\[1em]
  {\em Autor:} & \autor \\[1em]
  {\em Studijní program:} & \program \\
  {\em Obor:} & \obor \\
  {\em Druh práce:} & \druh \\[1em]
  {\em Vedoucí práce:} & \odstavec{\delka}{\vedouci\\ \pracovisteVed} \\
  %{\em Konzultant:} & -- %\odstavec{\delka}{\konzultant \\ \pracovisteKonz}  % VYMAŽTE text "-- %" v~případě, že jste neměli konzultanta
 \\[1em]
  %\multicolumn{2}{l}{\odstavec{\textwidth}{{\em Abstrakt:} ~ \abstrCZ  }} \\[1em]
  %\multicolumn{2}{l}{\odstavec{\delka}{{\em Abstrakt:} ~ \abstrCZ  }} \\[1em]
  {\em Abstrakt:} & \odstavec{\delka}{\abstrCZ} \\[2em]
  {\em Klíčová slova:} & \odstavec{\delka}{\klicova} \\[2em]

  {\em Title:} & ~\\
  \multicolumn{2}{l}{\odstavec{\textwidth}{\bf \nazeven}}\\[1em]
  {\em Author:} & \autor \\[1em]
  %\multicolumn{2}{l}{\odstavec{\textwidth}{{\em Abstract:} ~ \abstrEN  }} \\[1em]
		{\em Abstract:} & \odstavec{\delka}{\abstrEN} \\[1em]
  {\em Key words:} & \odstavec{\delka}{\keyword}
\end{tabular}

.\newpage
\thispagestyle{empty}
\chapter*{Prohlášení}
\noindent Prohlašuji, že jsem diplomovou práci vypracoval samostatně pod dohledem své\-ho ve\-dou\-cí\-ho
práce a~že jsem uvedl veškeré použité informační zdroje.\\

\noindent V Praze

\noindent \today \\

\begin{flushright}

....................................................\\
Jméno Autora
\end{flushright}

.\newpage 
\thispagestyle{empty}
\chapter*{Poděkování}
\noindent Děkuji vedoucímu práce za přínosné konzultace, pomoc a~odborné vedení této diplomové práce.

\noindent Děkuji také týmu ...
\newpage 
\thispagestyle{empty}